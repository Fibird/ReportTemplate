%%%%%%%%%%%%%%%%%%%%%%%%%%%%%%%%%%%%%%%%%
% University Laboratory Report mainly used in programming
% LaTeX Template
% Version 1.0 (12/9/16)
%
% Author: Chaoyang Liu
% E-mail: chaoyanglius@outlook.com 
% (https://chaoyangliu.cc)
%
% License:
% CC BY-NC-SA 3.0 (http://creativecommons.org/licenses/by-nc-sa/3.0/)
%
%%%%%%%%%%%%%%%%%%%%%%%%%%%%%%%%%%%%%%%%%

\documentclass[hyperref,UTF8]{ctexart}
\usepackage{listings} % Required for setting of code block
\usepackage[colorlinks,linkcolor=black]{hyperref}
\usepackage{color}
\usepackage{graphicx} % Required for the inclusion of images
\usepackage{varwidth} % Required for 
\usepackage{float}

\usepackage{graphicx} % Required for the inclusion of images
\usepackage{natbib} % Required to change bibliography style to APA
\usepackage{amsmath} % Required for some math elements 

\setlength\parindent{0pt} % Removes all indentation from paragraphs

\renewcommand{\labelenumi}{\alph{enumi}.} % Make numbering in the enumerate environment by letter rather than number (e.g. section 6)

%----------------------------------------------------------------------------------------
% SETTING OF CODE BLOCK
%----------------------------------------------------------------------------------------
\lstset{ % 代码高亮
	backgroundcolor=\color{white},   % choose the background color
	basicstyle=\footnotesize\ttfamily,        % size of fonts used for the code
	columns=fullflexible,
	numbers=left,                    % where to put the line-numbers; possible values are (none, left, right)
	numbersep=0.5em,		% how far the line-numbers are from the code
	breaklines=true,                 % automatic line breaking only at whitespace
	captionpos=t,                    % sets the caption-position to bottom
	tabsize=4,
	frame = single,
	framexleftmargin=2em,
	commentstyle=\color{green},    % comment style
	escapeinside={\%*}{*)},          % if you want to add LaTeX within your code
	keywordstyle=\color{blue},       % keyword style
	stringstyle=\color[rgb]{0.58,0,0.82}\ttfamily,     % string literal style
	rulecolor=\color{black},
	% identifierstyle=\color{red},
	language=c++,
	showtabs = false,
	showstringspaces = false,
	showspaces = false,
}

%----------------------------------------------------------------------------------------
% PDF INFORMATION
%----------------------------------------------------------------------------------------
% You need to set it in curly brace
% according to your information
\hypersetup{
	pdftitle={title of your document},
	pdfauthor={your name},
	pdfsubject={subject of your document},
	pdfkeywords={key words},
}

%----------------------------------------------------------------------------------------
%	DOCUMENT INFORMATION
%----------------------------------------------------------------------------------------

\title{这是实验报告的题目} % Title

\author{\kaishu 刘朝洋} % Author name

\date{\today} % Date for the report

\begin{document}

\maketitle % Insert the title, author and date

\begin{center}
\begin{tabular}{l r}
实验日期: & January 1, 2012 \\ % Date the experiment was performed
作者: & 刘朝洋 \\ % Partner names
指导老师: & 耿楠 % Instructor/supervisor
\end{tabular}
\end{center}

% If you wish to include an abstract, uncomment the lines below
% \begin{abstract}
% Abstract text
% \end{abstract}

%----------------------------------------------------------------------------------------
%	SECTION 1
%----------------------------------------------------------------------------------------

\section{实验目的}

这里主要是用于阐述本次实验的目的。

可以通过下列命令使用列表项:

\begin{enumerate}

\item 列表项1
\item 列表项2
\item 列表项3

\end{enumerate}
 
%----------------------------------------------------------------------------------------
%	SECTION 2
%----------------------------------------------------------------------------------------

\section{实验内容}

这里主要用于阐述本次实验的基本内容。

%----------------------------------------------------------------------------------------
%	SECTION 3
%----------------------------------------------------------------------------------------

\section{实验过程}

\subsection{使用代码块}

下面以Hello,word!程序为例:

\begin{lstlisting}
#include <iostream>

using namespace std;

int main()
{
    cout << "Hello world!" << endl;
    return 0;
}
\end{lstlisting}

\subsection{使用数学公式}

可以通过下面的方式使用数学公式:

\begin{equation}
AB^2 = BC^2 + AC^2
\end{equation}

如果要在句中使用数学符号或公式,可以使用$3 \times 3$。不同的数学符号有不同的命令,具体可以在参考该网站:\url{http://meta.math.stackexchange.com/questions/5020/mathjax-basic-tutorial-and-quick-reference}
%----------------------------------------------------------------------------------------
%	SECTION 4
%----------------------------------------------------------------------------------------

\section{结果与结论}

实验结果
\begin{figure}[h]
\begin{center}
\includegraphics[width=0.65\textwidth]{placeholder} % Include the image placeholder.png
\caption{Figure caption.}
\end{center}
\end{figure}

%----------------------------------------------------------------------------------------
%	SECTION 5
%----------------------------------------------------------------------------------------

\section{实验总结}

实验总结

%----------------------------------------------------------------------------------------
%	BIBLIOGRAPHY
%----------------------------------------------------------------------------------------

\bibliographystyle{apalike}

\bibliography{sample}

%----------------------------------------------------------------------------------------


\end{document}